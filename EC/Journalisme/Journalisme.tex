\documentclass[a4paper,11pt]{report}
\usepackage[french]{babel}
\usepackage[T1]{fontenc}
\usepackage[utf8]{inputenc}
\usepackage{lmodern}
\usepackage{microtype}
\usepackage{graphicx}
\usepackage{amsmath}
\usepackage{textcomp}

\title{EC}
\author{Florian Chatelier}
\date{12 Janvier 2019}
\begin{document}
\part{Écriture professionnelle : le modèle journalistique}
\tableofcontents
\chapter{Les agences de presse : historique}


Rappels :
\begin{itemize}
    \item \textbf{1631} : Théophraste Renaudot \underline{La Gazette}, qui comporte des nouvelles provenant de la Cour Française mais aussi des Cour étrangères grâce à un réseau de correspondants qui traduisent des feuilles d’information publié notamment en Allemagne et au pays Bas
    \item \textbf{1825} : 1ere ligne de chemin de faire en gb
    \item \textbf{1837} : Télégraphe 
    \item \textbf{1866} : 1er câble transatlantique
\end{itemize}
\section{1832 : Charles-Louis Havas}
Ouvre un bureau de traduction de journaux étrangers à destinations de diplomates, hommes politiques et journaliste français à Paris. \\
\textbf{1835} $\Rightarrow$ \textbf{"Agence Havas"} $\Rightarrow$ 1ere agence de presse internationale \\

Utilise chemins de fer et pigeons voyageurs puis dès 1845, télégraphe $\Rightarrow$ lui permet de dvp son réseau de correspondants à l’étranger (prends l’exemple de Théophraste Renaudot).

\textbf{Agence de presse} $\Rightarrow$ E B2B, organisme qui vend par abo des info à la presse, médias et autres orga pv ou publiques $\Rightarrow$ grossiste de l’info.
\section{1848 : l’AP (Associated Press) à NY}

MM année Havas accueille et forme 2 allemands exilés et les formes à Paris (Reuters et Wolff)
\begin{itemize}
    \item \textbf{1849} : \textbf{Wolff} crée l’agence \textbf{Wolff} à \textbf{Berlin}
    \item \textbf{1851} : \textbf{Reuters} crée l’agence \textbf{Reuters} à \textbf{Londres}
\end{itemize}

\section{Accord en 1859}
Du fait du coup élevé du télégraphe, les 4 agences vont se partager le monde \\
\begin{itemize}
    \item \textbf{Havas} : Europe méridional, les colories fr, l’Amérique du Sud 
    \item \textbf{AP} : NA
    \item \textbf{Reuters} : l’empire britannique, l’Extrême-Orient
    \item \textbf{Wolff} : Europe continentale (centrale, du Nord, de l’est)
\end{itemize}

\section{1917, création de l’agence Tass par les soviétiques}
\textbf{Années 30} : avec facilité des transmissions et montée du nationalisme, le partage du monde prend fin et les agences de presse entrent en concurrence directe.
\begin{itemize}
    \item Wolff est nationalisée et prend le nom de \textbf{Continental}. Organe du gouvernement hitlérien, cette agence disparaîtra avec lui.
    \item Agence Havas, nationalisé en \textbf{1940} par le gouv de \textbf{Vichy} et prend le nom d’\textbf{OFI (Office Français d’Information)}. 
    \item L'AFI est créée en parallèle à Londres \textbf{(Agence Française indépendante)} par la france libre avec le soutien du gouvernement britannique et de l'agence Reuters, 
    ce qui permets une diffusion quotidienne en Français de 10 000 mots vers l'Europe
\end{itemize}

\textit{Rmq} : Crée par Charles Louis Havas, échappe à la nationalisation par Vichy et est à l’origine de l’agence Havas aujourd’hui. \\
En 1944 à la libération, l’OFI est dénationalisé et devient l’AFP (Agence Française de Presse)

\chapter{L’AFP aujourd’hui}
(Source principale AFP – chiffre de 2016)
\section{Géographique}
\textbf{20 bureaux} dans 150 pays en 5 zones géographique et 5 centres régionaux.
\section{Équipes}
\begin{itemize}
    \item 3809 salariés dont 1503 journalistes (i.e qui ont leur carte de journaliste)
    \item Langues utilisées : le français, l’anglais, l’allemand, l’espagnol, le portugais, l’arabe.
    \item 3 clients : environ 4000 clients à travers le monde
    \item 35\% de progression entre 2005 à 2016
    \item Répartition : 74\% ; 26\% administration
    \item Tarif d’abonnement : jusqu’à 50 000 euros/mois
\end{itemize}

\chapter{Un double mot d’ordre : rapidité et rigueur}
La devise de Charles-Louis Havas était : « Vite et bien »
\section{Rapidité et exhaustivité}
L’importance logistique de l’AFP lui permet de délivrer l’information aux abonnés dans les délais extrêmement bref.
\section{Rigueur}

L’exigence de rapidité est soumise à un impératif supérieur, la fiabilité de l’information fourni. C’est ce qui permet à l’AFP de résister à la concurrence.\\
Chaque dépêche est vérifiée ou recoupée avant d’être envoyé au client.\\
Le statut de l’AFP voté par le parlement français en 1957 dégage l’AFP de toutes influences politiques ou commerciales.\\
Cette indépendance journalistique, permet à l’AFP de répondre à 3 engagements, vérité, impartialité.\\

\chapter{Les dépêches d'agence}
\section{Règles de présentation}
\begin{itemize}
    \item Ligne de codes
    \item mots clefs
    \item  titre chapeau : résume l’essentiel du contenu en une publication
    \item corps de la dépêche : lieu ou rattachement + date + information + source
    \item signature : rédacteur + relecteur (qui valide la dépêche)
\end{itemize}
 
\section{Règles de structure}
\begin{itemize}
    \item La « pyramide inversé » 
    \begin{itemize}
        \item 1er§ : le plus important  réponses aux 4 ou 5 W : Who ? What ? Where ? When ? (+Why ?)
        \item 2ème : l’analyse $\Rightarrow$ réponses aux questions complémentaires : (Why ?) How ?
        \item -§sq : les détails
    \end{itemize}
   \item La « loi de proximité temporelle » : $\Rightarrow$ Pas d’ordre chronologique mais du plus proche (présent) vers le plus éloigné (passé ou futur)
\end{itemize}

\section{Règles d’écriture}
\begin{enumerate}
    \item  Une recherche de la concision $\Rightarrow$ Un max d’info dans un minimum de mot.
    \item  Un vocabulaire simple et précis $\Rightarrow$ Pas de termes pauvres ou vagues, pas de phrases creuse, pas de clichés et de phrases toutes faites.
    \item La syntaxe simple et fluide $\Rightarrow$ Pas d’inversion d’interrogation incise ; voie active et tournure affirmative, 1 information par phrase.
    \item Les temps privilégié sont le passé composé et le présent : $\Rightarrow$ Pas de passé simple ni de présent de narration qui sont des temps employés dans des textes narratifs et non informatifs ; en revanche, le conditionnel est employé pour une information encore hypothétique..
    \item  … pas d’utilisation de la première personne, pas de terme subjectif pas de question rhétorique, métaphore, anaphore, figure de style $\Rightarrow$
\end{enumerate}

\textit{Ex:} France-mer-pollution-justice \\
Procès du naufrage de l’Erika : Total coupable Paris, 27 septembre (AFP). La cour de Cassation vient de confirmer la responsabilité pénale et civile des principaux acteurs du naufrage de l’Erika, dont Total.
\end{document}