\documentclass[a4paper,11pt]{report}
\usepackage[french]{babel}
\usepackage[T1]{fontenc}
\usepackage[utf8]{inputenc}
\usepackage{lmodern}
\usepackage{microtype}

\title{Expression et Communication}
\author{Florian Chatelier}
\date{14 Octobre 2018}

\begin{document}
\part{Modélisation de la communication}
\tableofcontents
\chapter{Le schéma de la communication}
\section{Contexte historique}
A partir de la maîtrise de l'électricité et de l'invention du télégraphe (1837) ce sont développés des moyens de communication moderne et les masses médias.
Au XXe siècle sont nés aux US, les sciences de la communication.

\subsection{Harold Lasswell}
Il a lancé un programme de recherche universitaire sur l'analyse des effets sur un schéma linéaire :
\begin{itemize}
    \item qui?
    \item dit quoi ?
    \item à qui ?
    \item comment ?
    \item avec quel effet ?
\end{itemize}

\subsection{Claude Shannon}
Mathématicien, ingénieur, électricien dans les télécommunications, pendant la WWII.
Avec \textbf{Warren Weaver}, ils publient une théorie mathématique sur la communication :
Le concept de bruit : tout ce qui peut entraîner \textit{la déperdition du message entre l'émission et la réception du message.}

\subsection{Norbert Weiner}
Il a introduit dans la modélisation des principes d'interaction et de feed-back \textit{la vision circulaire}

\section{Schéma de la communication}
\textbf{cf cours}

\subsection{L'émetteur}
\textbf{déf: L'émetteur émet le message}\\
\textit{rmq:} Quand un même message à plusieurs émetteurs $\Rightarrow$ messages collectifs
    ex: courrier de remerciement

\subsection{Le récepteur}
\textbf{déf: Le récepteur reçoit le message}\\
\textit{rmq:} on distingue le récepteur \textbf{intentionnel} ou \textbf{destinataire} du récepteur \textbf{non-intentionnel}
En fonction du nombre de destinataire, on distingue la communication :
\begin{itemize}
    \item intra-personnelle \textit{(l'émetteur est son propre destinataire)}
    \item interpersonnelle \textit{(l'émetteur d'adresse à un ou quelques destinataire(s))}
    \item collective \textit{(l'émetteur s'adresse à un groupe restreint ou défini de destinataires)}
    \item de masse 
\end{itemize}

\subsection{Le canal}
\textbf{déf: Le canal est le média utilisé par l'émetteur. \\ Les principaux canaux sont :}
\begin{itemize}
    \item la presse écrite
    \item le courrier postal
    \item le livre
    \item l'affichage
    \item le téléphone
    \item le cinéma
    \item la radio
    \item la télévision
    \item internet \\
\end{itemize}
\textit{rmq:} 
\begin{enumerate}
    \item dans la communication \textit{in praesentia} (émetteur et destinataires sont dans le même temps et même lieu), $\Rightarrow$ un canal n'est pas nécessaire
    \item dans la communication \textit{in absentia}, on distingue 3 cas :
    \begin{itemize}
        \item émetteur et destinataires ne sont pas dans le même lieu
        \item émetteur et destinataires ne sont pas dans le même temps
        \item émetteur et destinataires sont séparés dans l'espace et le temps
    \end{itemize}
\end{enumerate}

\subsection{Le code}
\textbf{déf: Le code est le moyen d'expression utilisé par l'émetteur pour encoder son message qui sera ensuite décodé par le récepteur}
\textit{rmq:} Considérant que l'Homme privilégie la vue et l'ouïe pour communiquer, on distingue 4 grandes catégories de code:
\begin{itemize}
    \item visuel-verbal \\
    ex: Texte écrit
    \item visuel-non verbal \\
    ex: Peinture
    \item auditif-verbal \\
    ex: Discours
    \item auditif-non verbal \\
    ex: Musique
\end{itemize}

\subsection{Le message}
\textbf{déf: Le message est le contenu intentionnel de l'émetteur, qu'il soit explicite ou implicite}
\textit{rmq:} L'émetteur délivre parfois aussi des messages non-intentionnel qu'ils soient conscient ou inconscient.

\subsection{Le feed-back}
\textbf{déf: Le feed-back représente la façon dont le récepteur réagit au message qu'il reçoit; il permet entre outre de capter le degré de fidélité entre l'émission et la réception du message.}
\textbf{Le feed-back peut être immédiat} (quand l'émetteur et récepteur sont dans la même temporalité) \textbf{ou différé}.

\subsection{Le référent}
\textbf{déf: Le référent est l'environnement du message, la réalité extérieure dans laquelle il est émis}
\textit{rmq:} On distingue :
\begin{itemize}
    \item \textbf{le référent contextuel :} il consiste à déterminer où et à partir de quoi un message est émis
    \item \textbf{le référent culturel :} présence allusive dans le message, considéré comme partagé entre l'émetteur et le récepteur
\end{itemize}
\newpage
\section{Les fonctions du langage}
\textbf{Roman Jakobson} a défini 6 fonctions du langage dans ses \textit{Essais de linguistique générale} - 1963
\begin{itemize}
    \item fonction référentielle
    \item fonction expressive
    \item fonction conative 
    \item fonction poétique
    \item fonction métalinguistique
    \item fonction phatique
\end{itemize}

\subsection{Définitions}
\subsubsection{La fonction référentielle}
L'émetteur l'utilise quand son message délivre des informations, des explications qui renvoient à  la réalité extérieure objective.
\textit{ex: Paul est avocat.}

\subsubsection{La fonction expressive}
L'émetteur l'utilise quand il parle de lui-même : ses sentiments, réactions, émotions, jugements.
\textit{ex: Quel avocat brillant, ce Paul !}

\subsubsection{La fonction conative}
L'émetteur l'utilise quand il veut exercer une influence sur son destinataire : l'émouvoir, le convaincre, l'inciter à modifier son comportement
\textit{ex: Ne trouves-tu pas que Paul est brillant ?}

\subsubsection{La fonction poétique}
L'émetteur l'utilise quand il fait un travail particulier sur le code, qu'il s'écarte de la norme.
\textit{ex: Paul est A-VO-CAT.}

\subsubsection{La fonction métalinguistique}
L'émetteur l'utilise quand son message porte sur le message lui-même ou le code qu'il emploie.
\textit{ex: Paul est un nom propre.}

\subsubsection{La fonction phatique (ou de contact)}
L'émetteur l'utilise quand il établit le contact avec le récepteur, et à chaque fois qu'il le maintient jusqu'à la cloture de la communication.
\textit{ex: Bonjour Paul, ça va ?}
La fonction phatique $\Rightarrow$ fonction transversale

Être un bon communicant, c'est faire un bon usage de la fonction phatique, celle-çi est protéiforme (d'aspect changeant).
\begin{enumerate}
    \item Ouvrir la communication, établir un contact \\
    \textit{ex: Cher Paul, etc..}
    \item Créer un lien avec le récepteur \\
    \textit{ex: regard, sourire, gestuelle}
    \item Faciliter la réception du message \\
    \textit{ex: mise en page claire et attrayante, orthographe correcte, bonne diction}
    \item Relancer l'attention de son récepteur \\
    \textit{ex: trait d'humour, anecdotes}
    \item Fermer la communication \\
    \textit{ex: Signature, "Merci"}
\end{enumerate}


\subsection{Remarques complémentaires}
\begin{enumerate}
    \item Les fonctions se rapportent toujours à l'émetteur. \\
    \textit{ex: Paul n'aime pas le cinéma}
    \item Un même message peut combiner plusieurs fonctions:
    \begin{itemize}
        \item l'une des fonctions sera dominante
        \item les autres des fonctions secondaires seront au service de la 1ère
    \end{itemize}
    \item La fonction dominante est dans la majorité des messages professionnels et est soit \textbf{référentielle}, soit \textbf{conative} ce qui permet de classer ces messages en deux grandes catégories:
    \begin{itemize}
        \item messages informatifs ou explicatifs \\
        \textit{ex: La notice d'utilisation, la rapport d'activité, la réunion d'information}
        \item messages argumentatifs \\
        \textit{ex: lettre de motivation, entretien d'embauche, le rapport critique, l'article d'opinion, la publicité, la réunion de travail, le débat}
    \end{itemize}
\end{enumerate}

\end{document}