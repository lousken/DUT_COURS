\documentclass[a4paper,11pt]{report}
\usepackage[french]{babel}
\usepackage[T1]{fontenc}
\usepackage[utf8]{inputenc}
\usepackage{lmodern}
\usepackage{microtype}
\usepackage{graphicx}
\usepackage{amsmath}
\usepackage{textcomp}

\title{EGOD}
\author{Florian Chatelier}
\date{09 Janvier 2019}
\begin{document}
\part{Droit de l'informatique}
\tableofcontents
\chapter{L'informatique et les libertés individuelles}
\section{Nos libertés individuelles à l'ère technologique}
Nos 4 libertés fondamentales : \\

\textbf{Notre droit à l’anonymat dans nos déplacements, question}
\begin{itemize}
    \item De la géolocalisation
    \item De la vidéo protection :
    \begin{itemize}
        \item Filment : la voie publique et les lieux ouverts au public et sont soumis aux dispositions du code de la sécurité intérieure.        
    \end{itemize}
    \item De la vidéo surveillance :
    \begin{itemize}
        \item Concernent : lieux non ouverts au public (bureaux ou réserves des magasins) et soumis aux dispositions de la loi « Informatique et Libertés ». 
    \end{itemize}
    \item \textbf{Droit à la préservation de notre identité}
    \item \textbf{Droit à la transparence}
    \item \textbf{Droit à la vie privée}
    \begin{itemize}
        \item La vie privée, en fait il faut pour être précis dire plutôt « le droit à l’intimité de la vie privée » fait partie des droits civils. \\
        vie privée $\Rightarrow$ pas déf pour éviter de limiter la protection aux seules prévisions légales. \\

    \end{itemize}
\end{itemize}


\textbf{Définition biométrie CNIL :}
La biométrie regroupe l’ensemble des techniques informatiques permettant de reconnaître automatiquement un individu à partir de ses caractéristiques physiques, biologiques, voire comportementales.
Les données biométriques sont des données à caractère personnel car elles permettent d’identifier une personne. \\
\textit{ex: Amazon Recognition} \\
Pays déployé biométrie $\Rightarrow$ Chine.


\section{Le protecteur de nos libertés : la CNIL}

\textbf{Commission Nationale de l’Informatique et des Libertés} \\
Créée en \textbf{1978} \\
Mission générale de la CNIL :
Veiller à ce que l’info soit au service du citoyen et qu’elle ne porte atteinte ni à l’identité humaine, ni aux droits de l’homme, ni à la vie privée, ni aux libertés individuelles ou publiques.

La CNIL est une \textbf{Autorité Administrative Indépendante} car :
\begin{itemize}
    \item statut assure son indépendance vis-à-vis de l’État
    \item représentée par une personne physique, son président (Isabelle Falque Pierrotin) \\
\end{itemize}

Composition de la CNIL : 
La commission se compose de \textbf{18 membres}\\

Moyens de la CNIL :
\begin{itemize}
    \item 17 161 536 € de budget en 2017
    \item 198 agents à fin 2017\\
\end{itemize}

Missions de la CNIL : \\
\begin{itemize}
    \item \textbf{Informer} \\
    La CNIL informe les personnes droits et obligations. Elle propose au gouvernement mesures de nature à adapter la protection des libertés et de la vie privée à l’évolution des techniques.\\
    \textit{Rmq:} Pour création E, CNIL met en avant le principe du \textbf{« privacy by design »}: prise en compte de la loi Informatique et Libertés et du RGPD dès la conception du produit ou service.\\
    \item \textbf{Recenser les fichiers} \\
    La CNIL exerce, pour le compte des citoyens qui souhaitent, l’accès aux fichiers intéressant la sûreté de l’État, la défense et la sécurité publique notamment ceux des Renseignements Généraux et de Police Judiciaire.
    2017 : 4039 D d’accès.\\
    \item \textbf{Garantir le droit d’accès}\\
    \item \textbf{Contrôler} \\
    La CNIL vérifie que la loi est respectée en contrôlant les applications informatiques.\\
    La Commission use de ses pouvoirs de vérification et d’investigation :
    \begin{itemize}
        \item Pour instruire les plaintes
        \item Pour disposer d’une meilleure connaissance de certains fichiers
        \item Pour mieux apprécier les conséquences du recours à l’informatique dans certains secteurs
    \end{itemize}
    \textit{ex: Sanction Google 2014 (Non application du droit à l’oubli)}\\ \\
    A la fin 2017 :
    \begin{itemize}
        \item 341 contrôles
        \item 79 mises en demeure (= ultimatum pour solutionner un problème)
        \item 14 sanctions \\
    \end{itemize}
    Quelques sanction 2018 :
    \begin{itemize}
        \item Janvier : Darty, 100 000€ (Insuffisance à la sécurisation des données des clients ayant eu recours au SAV en ligne)
    \end{itemize}
    \item \textbf{Réglementer} \\
    La CNIL établit des normes simplifiées pour que les traitements les + courants et les - dangereux pour les libertés fassent l’objet de formalités allégées.\\
\end{itemize}

RGPD : \textbf{25 mai 2018} \\
\textbf{RGPD} : Règlement G sur la Protection des Données \\
\textbf{GDPR} : General Data Protection Regulation\\
But : \textbf{unifier les approches de la réglementation des données de tous les États membres}, en veillant à ce que \textbf{toutes les lois sur la protection des données soient appliquées de la même manière} dans l’ensemble de ces pays.\\

\textbf{OTT} : over the top (en contournant le fournisseur d’accès internet, exemple : Netflix) \\
\textbf{Marché volatile} : variation du cours de la bourse importante\\

Avant, le \textbf{CIL = Correspondant Informatique et Libertés} \\
Après, le \textbf{DPO = Data Protection Officer = délégué à la protection des données}\\

Depuis 2005, E, établissements publics, associations peuvent désigner un Correspondant Informatique et Libertés. \\
\begin{itemize}
    \item Fin 2017 : 18 802
    \item Nombres de CIL en France fin 2017 : 5107
\end{itemize}

Avantage principal :
\begin{itemize}
    \item Allègement des formalités auprès de la CNIL.
\end{itemize}

Principales caractéristiques du CIL (DPO) :
\begin{itemize}
    \item Il doit être indépendant : Pas responsable des traitements informatiques\\
    \item Il peut assurer d’autres fonctions : Direction juridique, service qualité\\
    \item Il doit avoir les qualifications nécessaires\\
    \item Il peut s’agir d’une personne interne ou externe à l’organisation\\
    \item Il peut être désigné par plusieurs organisations \\
\end{itemize}

DPO $\Rightarrow$ super CIL\\
Pour les entreprises qui ont un CIL, ce dernier est légitime à occuper cette fonction.\\
DPO $\Rightarrow$ \textbf{périmètre plus large.}\\
« Le CIL arrête la liste des traitements et s’assure de leur conformité. Le DPO va devoir, lui, \textbf{savoir évaluer les risques}».\\

\textbf{Double culture de la gestion des risques et de la conformité mais aussi des compétences en IT et en sécurité.}

\chapter{La Cyber conflictualité | La Cyber Guerre}
\section{Notions Cyberespace, Cyber conflictualité}
\textbf{Cyberespace} 
Selon l’Agence Nationale de la Sécurité des Systèmes d’Information (ANSSI),
Le cyberespace est l’espace de communication constitué par l’interconnexion mondiale d’équipements de traitement automatisé de données numérisées.
Caractérisé par une approche sédimentaire en trois couches :
\begin{itemize}
    \item Physique ou Matérielle \\ 
    Regroupe les appareils d’extrémité (ordinateurs, box de fournisseur d’accès internet, disques durs, carte de crédit, distributeur de billet de banque…) ainsi que les infrastructures réseau.
    \item Logique ou Logicielle \\
    Regroupe les dispositifs de codage et de programmation qu’utilisent les machines.\\
    La pensée humaine est transformée en information via des interfaces homme-machine et des protocoles permettant la communication entre machines au sein d’un réseau, afin qu’elles puissent se transmettre l’information.\\
    \item Sémantique, Cognitive ou informationnelle \\
    Regroupent les données ou métadonnées qui sont transportées par le réseau. \\
    Métadonnées $\Rightarrow$ données de masse\\
    Elles peuvent déterminer les goûts et influencer ou favoriser la prise de décision des conso.\\
\end{itemize}


\textbf{Métadonnées} :  données qui en décrivent d’autres.

\textit{Ex : L’auteur, la date de création, la date de modification et la taille du fichier}

\textbf{Cyberconflit} : Echange d’actions qui utilisent des outils cyber ou des agression cyber pour aider chaque partie à obtenir un avantage sur l’autre dans le cadre de la rivalité qui les oppose.
Caractéristiques ?
\begin{itemize}
    \item Les attaques peuvent être l’œuvre d’un Etat, d’une entreprise, de militants, de particuliers et avoir diverses cibles.
    \item Il s’agit d’une guerre asymétrique :
    \begin{itemize}
        \item Faibles coûts
        \item Accessible par tous
        \item Avantage offensif
    \end{itemize}
\end{itemize}

\textit{rmq} : l’attaquant est en situation de force car coût de l’attaque $\Rightarrow$ faible alors que se défendre $\Rightarrow$ coup élevé

\section{Les types d'attaques}

Les conflictualités dans le cyber peuvent être caractérisées par une ou plusieurs combinaisons :
\begin{itemize}
    \item D’actions cyber espionnage de recherches d’informations \\
    \item D'actions correspondant à des cyberattaques :
    \begin{itemize}
        \item D’actions de perturbation \\
        \item De destruction\\
        \item De prise de contrôle à distance de système informatique\\
        \item D’actions de propagande et de manipulation de l’information assimilable à de la perturbation pour modifier l’opinion\\
    \end{itemize}
\end{itemize}


\section{Les conséquences des attaques}
Conséquence d’une attaque ?
\begin{itemize}
    \item Physique pour la population : paralysie des services médicaux, risque accrus d’accident de la route suite au défaut de signalisation…
    \item Psychologiques
    \item Environnementales : effets sur la maintenance des sites nucléaires, dysfonctionnement des égouts…
    \item Économiques : pertes industrielles, financières, sociales…
    \item Politiques : émeutes
    \item Liées au cumul et à la combinaison des circonstances ci-dessus
\end{itemize}
 
\section{La réponse des États}
\textbf{États Unis}
\begin{itemize}
    \item Le cybercommand\\
    \item Annoncée en 2007\\
    \item Création effective en 2009\\
    \item Organisation d’opérations de simulation d’attaques \\
\end{itemize}
  
\textbf{France}
\begin{itemize}
    \item L’ANSSI : Agence Nationale de la Sécurité des Système d’Information $\Rightarrow$ défense
    \item Crée en juillet 2009
    \item Missions ?
    \begin{itemize}
        \item Détecter attaques
        \item Prévenir menace
        \item Conseiller adm et OIV
        \item Informer régulièrement E et grand public
    \end{itemize}
    \item Force Offensive, mise en œuvre Décembre 2017
    \item D’ici 2019, 2600 combattants numériques
    \item 4400 réservistes
    \item Budget : 250 millions d’€ par an
    \item Objectif offensif
    \item Data Dome
\end{itemize}
\end{document}