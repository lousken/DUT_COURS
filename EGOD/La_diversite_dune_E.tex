\documentclass[a4paper,11pt]{report}
\usepackage[french]{babel}
\usepackage[T1]{fontenc}
\usepackage[utf8]{inputenc}
\usepackage{lmodern}
\usepackage{microtype}
\usepackage{graphicx}
\usepackage{amsmath}
\usepackage{textcomp}

\title{EGOD}
\author{Florian Chatelier}
\date{14 Octobre 2018}
\begin{document}
\tableofcontents
\part{Introduction au monde de l'E}
\chapter{La diversité d'une E}
\section{Les secteurs d'activité d'une E}
\begin{itemize}
    \item \textbf{Secteur primaire} (agriculture)
    \item \textbf{Secteur secondaire} (industrie)
    \item \textbf{Secteur tertiaire} (services)\\ 
    \textit{ ex: bancaire, informatique} 
\end{itemize}
\textbf{Information Technology:}
\begin{itemize}
    \item Technologie qui désigne l'usage des ordinateurs, stockage, réseaux et processus.
    \item Pour créer, traiter, stocker, sécuriser, et échanger, toutes sorties de données électronique.\\
\end{itemize}
\textbf{Start Up:}
Entreprise souvent très récente, qui évolue dans un environnement technologique très mouvant et qui connaît une forte croissance de son chiffre d'affaires et de ses effectifs.
\\
On peut ranger les Start Up par type:
\begin{itemize}
    \item \textbf{Fintech} (Finance et Technologie) \\
    \textit{ex: Leetchi, N26}
    \item \textbf{Biotech} (Santé et Technologie) \\
    \textit{ex: Imprimante 3D pour prothèses}
    \item \textbf{Foodtech} (Alimentaire et Technologie) \\
    \textit{ex: Take Eat Easy}
\end{itemize}
\newpage
\section{Objectifs d'une E ?}
\subsection{Objectifs économiques}
Conquête des parts de marché:
\begin{itemize}
    \item Calcul : $ \frac{ventes de l'E d'un produit}{ventes total d'un produit} $ 
    \item Peut être calculée en volume (qtt) ou valeur (\texteuro). \\
\end{itemize}
Distinction fondamentale : 
\begin{itemize}
    \item CA = montant des ventes, recettes, revenus = qttes vendues * \texteuro  de ventes
    \item Résultat = marge = CA - toutes les dépenses \\
    Si résultat positif $\Rightarrow$ Bénéfice = Gain \\
    Si résultat négatif $\Rightarrow$ Perte \\
\end{itemize}
\textbf{Remarque: } Lorsqu'une E a de grandes difficultés financières, le tribunal de commerce prononce la \textbf{liquidation judiciaire.} 

\subsection{Objectifs non-économiques}
\begin{itemize}
    \item Financement de projets sportifs ou culturels \\
    \textit{ex: Orange qui sponsorise le Tour de France}
    \item Prise en compte du dvp durable \\
    \textit{ex: Green IT}
    \item prise en considération du personnel
\end{itemize}
\subsubsection{E et groupes d'individus}
\subsubsection{Culture d'E}
Ensemble des \textbf{attitudes communes} à la plupart des membres d'une E et les principales \textbf{valeurs partagées.}
\subsubsection{E et un individu}
\textsc{Pyramide de Maslow}\\
\textbf{Besoin de réalisation} $\Rightarrow$ le salarié souhaite s'épanouir, réaliser, créer\\
\textbf{Besoin d'estime} $\Rightarrow$ le salarié souhaite être reconnu (prime, augmentation)\\
\textbf{Besoin d'appartenance} $\Rightarrow$ le salarié a la possibilité de faire parti d'un tout, d'une aventure\\
\textbf{Besoin de sécurité} $\Rightarrow$ le salarié doit pouvoir se protéger au sein de l'E(CDI)\\
\textbf{Besoin physiologiques} $\Rightarrow$ le salarié doit percevoir une rémunération qui va lui permettre de vivre\\

\chapter{La création de son E}
\section{L'idée ?}
\textit{ex: Glowee, Mars 2016 :} Start up FR qui veut remplacer électricité par bactéries\\
\textit{ex: Fairphone, Juillet 2017 :} E hollandaise qui veut révolutionner smartphones
\section{E commerciale ou industrielle ?}
\textbf{E commerciale} : achat de marchandise pour revente \textit{ex: Fnac, Auchan}
\textbf{E industrielle} : achat de marchandise pour revente 
\begin{itemize}
    \item achat de mat 1ère
    \item transformation (= fabrication)
    \item vente du produit fini
\end{itemize} 
\textit{ex: Apple, Renault, Atol} \\
La fabrication peut être sous-traitée. \\
\textit{ex: Apple et Foxconn}
\section{L'environnement de l'E}
\subsection{Environnement Global}
6 grandes catégories : \textbf{PESTEL}
\begin{itemize}
    \item pol
    \item éco
    \item socio
    \item techno
    \item écolo
    \item légales
\end{itemize}
\subsection{Environnement Concurrentiel}
\textsc{M.PORTER}\\
cf cours.
Les 5 facteurs de l'Environnement Concurrentiel permettent à l'E de situer son marché. \\
E analyse opportunités et menaces.
\section{Créateurs de l'E}
\begin{itemize}
    \item \textbf{SARL (société à responsabilité limitée)}  $\Rightarrow$ \textbf{associés} qui détiennent part sociétales
    \item \textbf{SA (société anonyme)} $\Rightarrow$ \textbf{actionnaires} qui détiennent des actions de l'E (titre financier) $\Rightarrow$ possèdent une part de l'E
\end{itemize}
\textsc{Apport en numéraire (argent) + Apport en nature (biens) = CAPITAL DE L'E} \\
A la création de l'E, capital = ce que l'E possède \\
Au cours de la vie de l'E elle peut avoir besoin \textbf{d'argent frais}, elle peut réaliser une hausse du capital \textbf{(= une levée de fonds)} \\
Si elle rencontre des difficultés, elle peut procéder à une baisse du capital
\section{Actionnaires}
\subsection{Définition}
Un actionnaire est une personne qui détient un pourcentage du capital de l'E.
\subsection{Droits}
Droits de chaque actionnaire : 
\begin{itemize}
    \item droit à l'Information
    \item droit de vote aux AGO, AGE 
    \textit{Rmq :} Dans la majorité des cas : 1 action = 1 vote
    \item droit de percevoir les dividendes (sur les bénéfices d'une E)
\end{itemize}
\subsection{Obligations}
Obligations de chaque actionnaire :
\begin{itemize}
    \item Volonté de participer à la vie de l'E
    \item Contribuer aux pertes si nécessaire
\end{itemize}
    \textit{Rmq :} Si un actionnaire détient 4\% du capital d'une E alors :
    \begin{itemize}
        \item il est propriétaire de celle-çi à hauteur de 4\%
        \item il dispose de 4\% des droits de vote
        \item il percevra 4\% du profit
        \item il contribuera aux pertes à hauteur de 4\% 
    \end{itemize} 

    \textbf{Business Angel (actionnaire particulier)}
    BA = personne physique qui investit une part de son patrimoine dans une E innovante à potentiel (start up)\\
    
\end{document}